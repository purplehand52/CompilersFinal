\chapter{Lexemes}\label{ch:Lexemes}

                                                                                                                                                                                  
\section{Comments}                                                                                                                                                                  
\begin{itemize}                                                                                                                                                                                
    \item \$ \texttt{this is a valid comment} \$                                                                                                                                                        
    \item \$\\  \texttt{                                                                                                                                                                               
    this is a valid \\                                                                                                                                                                         
    multiline comment\\    }                                                                                                                                                                    
    \$                                                                                                                                                                                         
\end{itemize}                                                                                                                                                                                  

\section{Whitespaces}                                                                                                                                                                
We ignore all whitespaces in the \texttt{\textbackslash output} section; although we consider newlines in the \texttt{\textbackslash init} and \texttt{\textbackslash main} sections. 

\section{Reserved Keywords}
Some of the keywords used in the language are listed down below:
\begin{itemize}
    \item \texttt{registers}
    \item \texttt{quantum}
    \item \texttt{classical}
    \item \texttt{iters}
    \item \texttt{set}
    \item \texttt{states} 
    \item \texttt{block}
    \item \texttt{for}
    \item \texttt{for\_lex}
    \item \texttt{for\_zip}
    \item \texttt{condition}
    \item \texttt{otherwise}
    \item \texttt{begin}
    \item \texttt{end}
    \item \texttt{output}
    \item \texttt{barrier}
    \item \texttt{X}, \texttt{Y}, \texttt{Z}, \texttt{H} (gate names)
\end{itemize}

Apart from the above keywords, we also include reserved keywords for datatypes like \texttt{int}, \texttt{list}, \texttt{float}, \texttt{complex}, \texttt{matrix} etc.

\section{Punctuations}
Some of the punctuations in our language include $\left[,\right]\,\{,\},(\,)\,;,\rightarrow$.

\section{Identifiers}
We again standard conventions for identifiers; they can start with \_ or a character from the alphabet, followed by an alphanumeric sequence.

\section{Operands}
All standard operands in most common \& general programming languages are available. Apart from these operands, we also use $\rightarrow$ indicating control from one qubit to another.
Similarly, we have ternaries to indicate a Classical-Register's control over a qubit. This is illustrated in the following section.
