\chapter{Structure}\label{ch:struct}


\section{\texttt{init} Section}
This section contains three necessary statements that provide the compiler with necessary information about qbits
\begin{lstlisting}
        #registers quantum = 3
        #registers classical = 2
        #iters = 1000
\end{lstlisting}
to set number of quantum registers, set number of classical registers and set number of iterations, respectively. \\

Optional statements include initializing states of quantum registers:
\begin{lstlisting}
    #set quantum states -> (2,3), (0,1), (5,5)
    #set classical states -> 0, 1
\end{lstlisting}
This sets the state of first quantum register as $2+3i$, second one as $i$ and third one as $5+5i$ and sets the state of first classical register as 0 and second one as 1.\\

This section will also have user defined code block definitions (refer to block definitions section)
\begin{lstlisting}                                                                                                                                                                             
    block (i,j,k) -> (p,q,r) {                                                                                                                                                                 
        $                                                                                                                                                                                      
            statement calls                                                                                                                                                                    
        $                                                                                                                                                                                      
    }                                                                                                                                                                                          
\end{lstlisting}                                                                                                                                                                               
                                                                                                                                                                                                   
\section{\texttt{main} Section}                                                                                                                                                              
    \textbackslash begin marks the beginning of the main section and \textbackslash end marks the end of the main section. This section includes                                               
    \begin{itemize}                                                                                                                                                                            
        \item calls to pre-defined or user-defined  \emph{blocks} (refer to block definitions section)                                                                                         
        \item \textbackslash barrier: It's a directive to the compiler to not combine blocks before and after \textbackslash barrier                                                           
        \item  \emph{measure} calls block that reads the value of a register and stores it in another register (both registers provided by the programmer)                                     
        \item condition-otherwise for conditional statements                                                                                                                                   
        \item  \emph{for},  \emph{for\_lex} and \emph{for\_zip} loops for repetitive calls to blocks.                                                                                          
    \end{itemize}     
        
\section{\texttt{output} Section}
    This section contains the features of any general programming language to help interpret the results of the quantum experiment further and better understand the results from 
    repeated runs of the circuit. \\
    This section mostly contains C-like features but syntactic sugar for certain operations and loops. We also support more datatypes for vector and array manipulation.
                                                                                                                                                                                                   
                                                                                                                                                                                                   
