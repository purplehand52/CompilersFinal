\chapter{Aim \& Program Flow}\label{ch:Program}

\section{DSL Objectives}
This program helps interested users define a Quantum-circuit, composed of:
\begin{itemize}
    \item Quantum-Registers
    \item Classical-Registers
    \item Quantum-Gates
    \item Measurement Boxes
\end{itemize}

Users can construct circuits of their choice using predefined gates and also define blocks of gates for further usage. Users can also simulate 
measurements and accordingly store them in necessary Classical-Registers. Users can also use Classical-Registers to condition the use of gates and 
control flow of the circuit. \\

Users have the option to manipulate their data collected in the Clasical-Registers over multiple iterations over the circuit with specified input 
qubits. Users can compare these values with their assumed distributions and describe error, variance and other parameters. Users can also do basic 
arithmetic operations on these output-vectors. \\

\section{Program}

The basic program consists of three sections:
\begin{itemize}
    \item \texttt{init}
    \item \texttt{main} denoted by \texttt{\textbackslash begin} and \texttt{\textbackslash end}
    \item \texttt{output}
\end{itemize}

Please find below a sample program that illustrates quantum teleportation.
\begin{lstlisting}[caption=Quantum Teleportation, label=code:SampleCode]
    \begin_init
        #registers quantum = 3
        #registers classical = 3
        #iters = 1000

        #set quantum states -> (1,0), (1,0), (1,0)
        #set classical states -> 0, 0, 0

        block BellGen(i, j) -> (j) [
            $ Start with Hadamard $
            H: i
            $ Apply C-X $
            X: i -> j
        ]
    \end_init

    \begin
        $ Generate Bell State $
        BellGen : (2, 3) -> (3)
        
        \barrier
        
        $ Init Transformations $
        X: 1 -> 2
        H: 1

        $ Measurements $
        measure: 1 -> 1
        measure: 2 -> 2

        $ Classical Controlled Ops $
        2 ? X : I
        1 ? Z : I

        $ Final Measurement $
        measure: 3 -> 3
    \end

    \begin_output
        $ Copy Output $
        int sat_alpha = 0;
        int sat_beta = 0;
        for i in (0:7)
        {
            condition(i%2 == 0)
            {
                sat_alpha = sat_alpha + c_output[i];
            }
            otherwise
            {
                sat_beta = sat_beta + c_output[i];
            }
        }
        
        $ Comparision $
        list final = [sat_alpha, sat_beta];
        list dist = [1000, 0];
        list c = diff(final, dist);
        echo(c); 

    \end_output
\end{lstlisting}