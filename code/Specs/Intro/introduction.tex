\chapter{Basic Quantum Computing}
\pagenumbering{arabic} \setcounter{page}{1}

\section{Background}

As computer science takes root in all aspects of our lives over the last few decades, computer scientist all over the world have begun to explore new paradigms.
One of these promising paradigms includes Quantum-Computing; using the postulates of Quantum-Mechanics to possibly achieve machines that are more powerful than conventional Turing-Complete machines.
Our DSL aims to provide a gateway to understand simple Quantum-Circuits and consequences of these circuits on a set of input quantum-bits, better known as qubits.

\section{Qubits \& Quantum-Gates}

In the classical computing paradigm, we use classical bits that deterministically take 0 or 1 as its value. Qubits on the other hand, considers a \emph{superposition} of 0 and 1.
This can be represented as follows:
\begin{align}
    \ket{\psi} &= \alpha\ket{0} + \beta\ket{1} \\
        &= \begin{pmatrix}\alpha\\\beta\end{pmatrix}
\end{align}

Classical-Gates like \textbf{AND}, \textbf{OR}, \textbf{NOT}, \textbf{XOR} perform operations on a certain set of classical bits. Similarly, Quantum-Gates 
perform certain transformations on the vector representing a qubit. As a result, all Quantum-Gates can be represented through unitary matrices statisfying the 
following property:
\begin{align}
    UU^{\dag} = U^{\dag}U = \mathds{1}
\end{align}
Unlike Classical-Gates, Quantum-Gates always have equal number of input-qubits and output-qubits, i.e., $fanin = fanout$.

\section{Single Qubit Gates}

Similar to a \textbf{NOT} gate in the classical computing paradigm, we have the \textbf{X} gate which serves as the quantum analgoue. We represent the \textbf{X} gate 
through the following $2\times2$ matrix as follows:
\begin{align}
    \mathbf{X} = \begin{bmatrix}
        0 & 1\\
        1 & 0
        \end{bmatrix}
\end{align}
When \textbf{X} acts on a state-vector $\ket{\psi} = \alpha\ket{0} + \beta\ket{1}$, it switches the coordinates of the vector as follows.
\begin{align}
    \mathbf{X}\ket{\psi} = \begin{bmatrix}
        0 & 1\\
        1 & 0
        \end{bmatrix}
        \begin{bmatrix}
            \alpha\\
            \beta
        \end{bmatrix} = \begin{bmatrix}
            \beta \\
            \alpha
        \end{bmatrix}
\end{align}
Similarly, you have the \textbf{Y}, \textbf{Z} and other associated rotation gates along different axes. One other single qubit gate that is commonly used is 
the \textbf{Hadamard} gate, represented as follows:
\begin{align}
    \mathbf{H} = \begin{bmatrix}
        \frac{1}{\sqrt{2}} & \frac{1}{\sqrt{2}}\\
        \frac{1}{\sqrt{2}} & \frac{-1}{\sqrt{2}}
        \end{bmatrix}
\end{align}

\section{Multiple Qubit Gates}

Consider a \textbf{CNOT}/\textbf{CX} quantum-gate. This gate takes two qubits as input: control-qubit and target-qubit. This gate acts on the target qubit 
only when the control qubit is in state $\ket{1}$. Its matrix representation is as follows:
\begin{align}
    \mathbf{CX} = \begin{bmatrix}
        1 & 0 & 0 & 0\\
        0 & 1 & 0 & 0\\
        0 & 0 & 0 & 1\\
        0 & 0 & 1 & 0
        \end{bmatrix}
\end{align}

Similarly one can define any \textbf{CA} quantum-gate which performs the \textbf{A} quantum-gate on the target-qubit based on the control-qubit. Similarly, 
we have doubly-controlled \textbf{NOT} gate also known as the \textbf{Toffoli} gate, represented as follows:
\begin{align}
    \mathbf{CCX} = \begin{bmatrix}
        1 & 0 & 0 & 0 & 0 & 0 & 0 & 0\\
        0 & 1 & 0 & 0 & 0 & 0 & 0 & 0\\
        0 & 0 & 1 & 0 & 0 & 0 & 0 & 0\\
        0 & 0 & 0 & 1 & 0 & 0 & 0 & 0\\
        0 & 0 & 0 & 0 & 1 & 0 & 0 & 0\\
        0 & 0 & 0 & 0 & 0 & 1 & 0 & 0\\
        0 & 0 & 0 & 0 & 0 & 0 & 0 & 1\\
        0 & 0 & 0 & 0 & 0 & 0 & 1 & 0\\
        \end{bmatrix}
\end{align}

\section{Measurement}

Upon measurement of any qubit $\ket{\psi} = \alpha\ket{0} + \beta\ket{1}$, we obtain a classical bit $0$ with probability $|\alpha|^2$ and $1$ with probabililty 
$|\beta|^2$. As always, we can measure a certain qubit out of a vector of qubits. For instance consider a 2-qubit state as follows:
\begin{align}
    \ket{\psi} = \alpha\ket{00} + \beta\ket{01} + \gamma\ket{10} + \delta\ket{11}
\end{align}
The probability of getting $0$ while measuring the first qubit would be $|\alpha|^2 + |\beta|^2$; whereas the probability of getting $0$ while measuring the second qubit 
would be $|\alpha|^2 + |\gamma|^2$. After measurement the qubit is lost and only the normalized version of the remaining qubits remain. Say we measure the second qubit and get 
$1$ as our measurement, our collapsed state is as follows:
\begin{align}
    \ket{\psi} \xrightarrow[\text{Obtain $1$}]{\text{Measure $2^{nd}$ Qubit}} \left(\frac{\beta\ket{0} + \delta\ket{1}}{\sqrt{|\beta|^2 + |\delta|^2}}\right)
\end{align}
% \section{About this Thesis}
% This is the thesis of \emph{Insert Full Name Here}, submitted as part of the requirements for the degree of MSc Computing: Software Technology at the School of Computing, Robert Gordon University, Scotland.

% A number of paragraphs detailing the main expectations of this body of work.


% \section{Chapter List}
% Provide a list of all the chapters within the  thesis and a brief summary of the content.

% \textbf{Chapter \ref{ch:Background}} Background Research. This chapter
% deals with $\ldots$.

% \textbf{Chapter \ref{ch:Design}} Design. This chapter
% deals with $\ldots$.

% \textbf{Chapter \ref{ch:Implementation}} Implementation. This chapter
% deals with $\ldots$.

% \textbf{Chapter \ref{ch:Evaluation}} Evaluation \& Testing. This chapter
% deals with $\ldots$.

% \textbf{Chapter \ref{ch:Conclusion}} Conclusion. The conclusions of the thesis are presented.

% \textbf{Chapter \ref{ch:usingLatex}} Using \LaTeX. This chapter
% deals with how to use the \LaTeX \space system. \textcolor{red}{ERASE THIS LINE ONCE YOU ARE DONE!}


% \section{Conclusion}
% A short conclusion summarising the chapter.