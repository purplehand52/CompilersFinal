\chapter{Statements}\label{ch:stmt}

\subsection{Save Command}
\begin{lstlisting}
     \save filename.csv
\end{lstlisting}
This command saves the output from main section of the code into an output csv file which can be used for further data manipulation

\subsection{Echo Command}
\begin{lstlisting}
     \echo
\end{lstlisting}
This prints the output from the main section of the code to the terminal

\section{Output Statements}
This section mainly involves mainpulating the data received from the main section. These are basic arithmetic operations and also allows graphical interpretation of the data received.\\
It allows basic programming constructs any other programming language can offer:

\begin{itemize}
    \item Declaration statements
    \begin{itemize}
        \item It follows C style syntax for variable declaration.
    \end{itemize}
        
    \item Expression statements
    \begin{itemize}
        \item This may involve using variables, carrying out arithmetic operations on them and assigning to other variables. 
    \end{itemize}
    \item Conditional statements
    \begin{itemize}
        \item It follows C-style conditional statements but with a different flavour:
        \begin{lstlisting}
            condition(x > 3){
                ...    
            }
            otherwise{
                ...
            }
        \end{lstlisting}
    \end{itemize}
    \item Loop statements
    \begin{itemize}
        \item Three kinds of for loops are provided:
        \begin{itemize}
            \item for: it initializes an integer (i in this case). The integer loops through values from start to end (exclusive). The step in values can be provided optionally. These parameters are provided as a tuple. This is followed by the body of statements enclosed by curly braces.
            \begin{lstlisting}
            for i in (start:end) 
            {
                X: i 
            }
            
            for i in (start:end:step) {
                Y: i 
            }
            \end{lstlisting}
                
            \item for\_lex: loops through all permutations of multiple integers in lexicographical order, in their respective ranges
            \begin{lstlisting}
            for_lex (i,j,k) in (0:5,5:10,10:15) {
                X: i -> j
                Y: j -> k
            }
            \end{lstlisting}
            
            \item for\_zip: loops through multiple variables simultaneously. 
            \begin{lstlisting}
            for_zip (i,j) in (0:3,9:4:-2) {
                X: i -> j
            }
            \end{lstlisting}
            Here (i,j) take values (0,9), (1,7), (2,5)
        \end{itemize}
    \end{itemize}
\end{itemize}

Some additional features for data-manipulation are provided: \textbf{(more to be added on the go)}
\begin{itemize}
    \item Difference or Sum of arrays
    \begin{lstlisting}
        x = diff(A,B)
        y = sum(A,B)
    \end{lstlisting}

    \item Standard deviation of a list of numbers
    \begin{lstlisting}
        a = std_dev(A)    
    \end{lstlisting}
    
    \item Variance of a list of numbers
    \begin{lstlisting}
        a = var(A)    
    \end{lstlisting}
    
\end{itemize}

\section{Operators}
\begin{enumerate}
    \item \textbf{X}:2 - Swap operation\\
    Here 2 is the target register and X is the operand
    \item \textbf{X}: 2 $->$ 3\\  Here 2 is the control register, 3 is target register and X is the operand.
    \item All other common operands in general programming languages are also included with similar operator precedence.
\end{enumerate}


\section{Binary If}
This statement consists of a classical register and a quantum operator with a ? in between. The quantum operation only takes place if the classical register contains a 1. 
\begin{lstlisting}
    2 ? X : 4->5 
\end{lstlisting}

\section{Measure}
This reads from a quantum register (left) and stores the observation to a classical register (right). An arrow is used as punctuation. The qubit ceases to exist after this operation.
\begin{lstlisting}
    measure: 4->1
\end{lstlisting}


\section{Block}
A collection of statements can be abstracted and reused like an operation. A block is analogous to a user defined function. 

\begin{itemize}
    \item \emph{Definition}: \\
    This consists of a call signature followed by a body of statements enclosed in square braces. The block-name consists of alphanumeric characters and underscore. It must start with an uppercase letter.

    \begin{lstlisting}
    block MyBlock: (a,b,c) [
        $ statements $
    ]
    \end{lstlisting}
    A user may also define a function with an arrow operator to make the control and target registers more explicit. 
    Note that a block may contain statements other than quantum operations, such as for.
    \begin{lstlisting}
    block MyBlock_2: (i,j,k) -> (p,q) [
        for t in (i..j)
        {
            Y: t->p
            k? X: q
        }
    ]
    \end{lstlisting}

  \item \emph{Call}:
    A block operator is fed arguments similar to operands in a quantum operation.
    \begin{lstlisting}
    MyBlock_2: (2,9,0) -> (0,1)
    \end{lstlisting}
\end{itemize}